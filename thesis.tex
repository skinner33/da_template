\RequirePackage[l2tabu, orthodox]{nag}
\documentclass[a4paper,11pt,twoside]{memoir}
\chapterstyle{veelo}

\usepackage{TUINFDA}

\usepackage{url}
\usepackage{graphicx}            			% Figures
\usepackage{verbatim}            			% Code-Environment
%\usepackage[lined,linesnumbered,algochapter]{algorithm2e} % Algorithm-Environment

\usepackage{pgf}
\usepackage{tikz}					% tikz graphics
\usetikzlibrary{arrows,automata,calc}


\usepackage{cite}       % Deutsche Bezeichnungen, Automatisches Zusammenfassen von Literaturstellen
\usepackage{varioref}  % Querverweise
% to use the german charset include cp850 for MS-DOS, ansinew for Windows and latin1 for Linux.
% \usepackage[latin1]{inputenc}

\usepackage{chapterfolder}
\usepackage[colorlinks]{hyperref}					% links in pdf
\usepackage{xspace}
\usepackage{siunitx}
\sisetup{
  per-mode=symbol,
  binary-units=true
}

\usepackage{subfig}
\newcommand{\subfigureautorefname}{\figureautorefname}

% improve typesetting
\usepackage[activate]{microtype}
% removes warnings due to usage of mircotype and siunitx, based on siunitx'
% redefinition of some style.
% check later if this has a bad effect on the typography
\usepackage{textcomp}
% for acronymns
\usepackage{acronym}
% for \pgf/tikz and \BibTeX
\usepackage{dtklogos}
% for \mathds{}
\usepackage{dsfont}
% for \mod{} (==modulo)
\usepackage{amsmath}
% fixes some quirks of amsmath
\usepackage{mathtools}
% for lemmata, theorems and proofs
\usepackage[thmmarks,amsmath,hyperref]{ntheorem}
% Algorithm package
\usepackage[ruled,noend,linesnumbered]{algorithm2e}
\usepackage{algpseudocode}
% for C style comments
\algrenewcomment[1]{\quad// #1}
% for 'fancy' \todo
\usepackage{todonotes}
% for notation page
\usepackage[refpage]{nomencl}
\makenomenclature
% right aligned labels in description-env
\usepackage{enumitem}
\usepackage{calc}
\SetLabelAlign{parright}{\parbox[t]{\labelwidth}{\raggedleft#1}}
% to exclude certian stuff in draft mode, to speed up compile
\usepackage{ifdraft}
% for subfigures
\usepackage{subfig}
% adds lines for tables with different spacing
\usepackage{booktabs}
% reduce the fontsize for captions
\usepackage[font=small,format=hang]{caption}

% noabbrev disables the use of fig. instead of figure
\usepackage[noabbrev,capitalise,nameinlink]{cleveref}
\crefname{assume}{assumption}{assumptions}
\crefname{lem}{lemma}{lemmas}
\crefname{thm}{theorem}{theorems}
\crefname{cor}{corollary}{corollaries}
\crefname{define}{definition}{definitions}

\usepackage{fixltx2e}



\thesistitle{Title of the Thesis}
\thesissubtitle{Optional Subtitle} % optional
\thesisdate{TT.MM.JJJJ}

% all titles and designations have to be gender-related!
\thesisdegree{Diplom-Ingenieurin}{Diplom-Ingenieurin}
\thesiscurriculum{Wirtschaftsinformatik}{Business Informatics} % your study
\thesisverfassung{Verfasserin} % Verfasser
\thesisauthor{Martina Musterfrau} % your name
\thesisauthoraddress{Musterplatz 1, 1111 Wien} % your address
\thesismatrikelno{0123456} % your registration number

\thesisbetreins{o.Univ.-Prof. Dipl.-Ing. Mag. Dr. Monika Musterprofessorin}
\thesisbetrzwei{Dr. Vorname Familienname}
\thesisbetrdrei{Dr. Vorname Familienname} % optional

% define page numbering styles
\makepagestyle{numberCorner}
\makeevenfoot{numberCorner}{\thepage}{}{}
\makeoddfoot{numberCorner}{}{}{\thepage}

% define custom macros for specific formats or names
%
% use \xspace at the to get smart spacing i.e. latex inserts a space when
% needed.
\newcommand{\uml}[1]{\texttt{#1}\xspace}
\newcommand{\cd}{\textsf{Class Diagram}\xspace}

%acronyms
\acrodef{TMR}{\emph{triple modular redundancy}}
%\DeclareAcronym{TMR}{
%	short = TMR,
%	long = triple modular redundancy,
%}

\begin{document}

\captionnamefont{\bfseries}

%%%%%%%%%%%%%%%%%%%%%%%%%%%%%%%%%%%%%%%%%
%%%   FRONTMATTER    %%%%%%%%%%%%%%%%%%%%
%%%%%%%%%%%%%%%%%%%%%%%%%%%%%%%%%%%%%%%%%
\frontmatter
\pagenumbering{roman}

%%%%%%%%%%%%%%%%%%%%%%%%%%%%%%%%%%%%%%%%%
%%%   TITLEPAGES    %%%%%%%%%%%%%%%%%%%%%
%%%%%%%%%%%%%%%%%%%%%%%%%%%%%%%%%%%%%%%%%

% the german title page is required as first page
\include{titlepage}

% an english translation may follow
\include{titlepage_en} % optional

%%%%%%%%%%%%%%%%%%%%%%%%%%%%%%%%%%%%%%%%%
%%%   ERKLAERUNG DER SELBSTAENDIGKEIT   %
%%%%%%%%%%%%%%%%%%%%%%%%%%%%%%%%%%%%%%%%%
\cleardoublepage
\selectlanguage{ngerman}
\input{chapters/erklaerung}
\selectlanguage{english}

%%%%%%%%%%%%%%%%%%%%%%%%%%%%%%%%%%%%%%%%%
%%%   ACKNOWLEDGEMENTS    %%%%%%%%%%%%%%%
%%%%%%%%%%%%%%%%%%%%%%%%%%%%%%%%%%%%%%%%%

% optional acknowledgements may be included in german or in english
%\input{chapters/danksagung} 		% optional
\input{chapters/acknowledgements}	% optional

%%%%%%%%%%%%%%%%%%%%%%%%%%%%%%%%%%%%%%%%%
%%%   ABSTARCT    %%%%%%%%%%%%%%%%%%%%%%%
%%%%%%%%%%%%%%%%%%%%%%%%%%%%%%%%%%%%%%%%%

\input{chapters/abstract_en}
\cleardoublepage
\selectlanguage{ngerman}
\input{chapters/abstract_de}
\selectlanguage{english}

%%%%%%%%%%%%%%%%%%%%%%%%%%%%%%%%%%%%%%%%%
%%%   CONTENTS    %%%%%%%%%%%%%%%%%%%%%%%
%%%%%%%%%%%%%%%%%%%%%%%%%%%%%%%%%%%%%%%%%
% uncomment to set document language to german (results in "Inhaltsverzeichnis",
% "Kapitel", "Abbildung", etc. instead of "Contents", "Chapter", and "Figure"),
% otherwise the document's language is english
%\selectlanguage{ngerman}

\setcounter{tocdepth}{1}

\cleardoublepage
\pagestyle{numberCorner}
\tableofcontents*

%%%%%%%%%%%%%%%%%%%%%%%%%%%%%%%%%%%%%%%%%
%%%   MAINMATTER    %%%%%%%%%%%%%%%%%%%%%
%%%%%%%%%%%%%%%%%%%%%%%%%%%%%%%%%%%%%%%%%

\mainmatter
% Include number also in subsubsections
\setsecnumdepth{subsubsection}

\pagenumbering{arabic}
\pagestyle{numberCorner}

%%%%%%%%%%%%%%%%%%%%%%%%%%%%%%%%%%%%%%%%%
\cfchapter{Introduction}{chapters}{introduction}
%%%%%%%%%%%%%%%%%%%%%%%%%%%%%%%%%%%%%%%%%

%%%%%%%%%%%%%%%%%%%%%%%%%%%%%%%%%%%%%%%%%
\cfchapter{Typographic Design}{chapters}{design}
%%%%%%%%%%%%%%%%%%%%%%%%%%%%%%%%%%%%%%%%%

%%%%%%%%%%%%%%%%%%%%%%%%%%%%%%%%%%%%%%%%%
\cfchapter{Bibliographic Issues}{chapters}{bibliographic}
%%%%%%%%%%%%%%%%%%%%%%%%%%%%%%%%%%%%%%%%%

%%%%%%%%%%%%%%%%%%%%%%%%%%%%%%%%%%%%%%%%%
\cfchapter{Updated Features}{chapters}{update}
%%%%%%%%%%%%%%%%%%%%%%%%%%%%%%%%%%%%%%%%%

%%%%%%%%%%%%%%%%%%%%%%%%%%%%%%%%%%%%%%%%%
%%% BACKMATTER %%%%%%%%%%%%%%%%%%%%%%%%%%
%%%%%%%%%%%%%%%%%%%%%%%%%%%%%%%%%%%%%%%%%

\appendix

\bibliographystyle{IEEEtranS}
\bibliography{references}

\printnomenclature

\end{document}
