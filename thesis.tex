\RequirePackage[l2tabu, orthodox]{nag}
\documentclass[a4paper,11pt,twoside]{memoir}
\chapterstyle{veelo}

\usepackage{TUINFDA}

\usepackage{url}
\usepackage{graphicx}            			% Figures
\usepackage{verbatim}            			% Code-Environment
%\usepackage[lined,linesnumbered,algochapter]{algorithm2e} % Algorithm-Environment

\usepackage{pgf}
\usepackage{tikz}					% tikz graphics
\usetikzlibrary{arrows,automata,calc}


\usepackage{cite}       % Deutsche Bezeichnungen, Automatisches Zusammenfassen von Literaturstellen
\usepackage{varioref}  % Querverweise
% to use the german charset include cp850 for MS-DOS, ansinew for Windows and latin1 for Linux.
% \usepackage[latin1]{inputenc}

\usepackage{chapterfolder}
\usepackage[colorlinks]{hyperref}					% links in pdf
\usepackage{xspace}
\usepackage{siunitx}
\sisetup{
  per-mode=symbol,
  binary-units=true
}

\usepackage{subfig}
\newcommand{\subfigureautorefname}{\figureautorefname}

% improve typesetting
\usepackage[activate]{microtype}
% removes warnings due to usage of mircotype and siunitx, based on siunitx'
% redefinition of some style.
% check later if this has a bad effect on the typography
\usepackage{textcomp}
% for acronymns
\usepackage{acronym}
% for \pgf/tikz and \BibTeX
\usepackage{dtklogos}
% for \mathds{}
\usepackage{dsfont}
% for \mod{} (==modulo)
\usepackage{amsmath}
% fixes some quirks of amsmath
\usepackage{mathtools}
% for lemmata, theorems and proofs
\usepackage[thmmarks,amsmath,hyperref]{ntheorem}
% Algorithm package
\usepackage[ruled,noend,linesnumbered]{algorithm2e}
\usepackage{algpseudocode}
% for C style comments
\algrenewcomment[1]{\quad// #1}
% for 'fancy' \todo
\usepackage{todonotes}
% for notation page
\usepackage[refpage]{nomencl}
\makenomenclature
% right aligned labels in description-env
\usepackage{enumitem}
\usepackage{calc}
\SetLabelAlign{parright}{\parbox[t]{\labelwidth}{\raggedleft#1}}
% to exclude certian stuff in draft mode, to speed up compile
\usepackage{ifdraft}
% for subfigures
\usepackage{subfig}
% adds lines for tables with different spacing
\usepackage{booktabs}
% reduce the fontsize for captions
\usepackage[font=small,format=hang]{caption}

% noabbrev disables the use of fig. instead of figure
\usepackage[noabbrev,capitalise,nameinlink]{cleveref}
\crefname{assume}{assumption}{assumptions}
\crefname{lem}{lemma}{lemmas}
\crefname{thm}{theorem}{theorems}
\crefname{cor}{corollary}{corollaries}
\crefname{define}{definition}{definitions}

\usepackage{fixltx2e}



\thesistitle{Title of the Thesis}
\thesissubtitle{Optional Subtitle} % optional
\thesisdate{TT.MM.JJJJ}

% all titles and designations have to be gender-related!
\thesisdegree{Diplom-Ingenieurin}{Diplom-Ingenieurin}
\thesiscurriculum{Wirtschaftsinformatik}{Business Informatics} % your study
\thesisverfassung{Verfasserin} % Verfasser
\thesisauthor{Martina Musterfrau} % your name
\thesisauthoraddress{Musterplatz 1, 1111 Wien} % your address
\thesismatrikelno{0123456} % your registration number

\thesisbetreins{o.Univ.-Prof. Dipl.-Ing. Mag. Dr. Monika Musterprofessorin}
\thesisbetrzwei{Dr. Vorname Familienname}
\thesisbetrdrei{Dr. Vorname Familienname} % optional

% define page numbering styles
\makepagestyle{numberCorner}
\makeevenfoot{numberCorner}{\thepage}{}{}
\makeoddfoot{numberCorner}{}{}{\thepage}

% define custom macros for specific formats or names
%
% use \xspace at the to get smart spacing i.e. latex inserts a space when
% needed.
\newcommand{\uml}[1]{\texttt{#1}\xspace}
\newcommand{\cd}{\textsf{Class Diagram}\xspace}

%acronyms
\acrodef{TMR}{\emph{triple modular redundancy}}
%\DeclareAcronym{TMR}{
%	short = TMR,
%	long = triple modular redundancy,
%}

\begin{document}

\captionnamefont{\bfseries}

%%%%%%%%%%%%%%%%%%%%%%%%%%%%%%%%%%%%%%%%%
%%%   FRONTMATTER    %%%%%%%%%%%%%%%%%%%%
%%%%%%%%%%%%%%%%%%%%%%%%%%%%%%%%%%%%%%%%%
\frontmatter
\pagenumbering{roman}

%%%%%%%%%%%%%%%%%%%%%%%%%%%%%%%%%%%%%%%%%
%%%   TITLEPAGES    %%%%%%%%%%%%%%%%%%%%%
%%%%%%%%%%%%%%%%%%%%%%%%%%%%%%%%%%%%%%%%%

% the german title page is required as first page
% $Id: titlepage.tex 1752 2010-03-20 11:07:02Z tkren $
%
% TU Wien - Faculty of Informatics
% thesis titlepage
%
% This titlepage is using the geometry package, see
% <http://www.ctan.org/macros/latex/contrib/geometry/geometry.pdf>
%
% For questions and comments send an email to
% Thomas Krennwallner <tkren@kr.tuwien.ac.at>
% or to Petra Brosch <brosch@big.tuwien.ac.at>
%

\selectlanguage{ngerman}

% setup page dimensions for titlepage
\newgeometry{left=2.4cm,right=2.4cm,bottom=2.5cm,top=2cm}

% force baselineskip and parindent
\newlength{\tmpbaselineskip}
\setlength{\tmpbaselineskip}{\baselineskip}
\setlength{\baselineskip}{13.6pt}
\newlength{\tmpparindent}
\setlength{\tmpparindent}{\parindent}
\setlength{\parindent}{17pt}

% first titlepage
\thispagestyle{tuinftitlepage}

%
% Kludge: for each titlepage set \pagenumbering to a different
% style. This is used to fix a problem with hyperref, because there
% are multiple "page 1" and hyperref hates that
%
\pagenumbering{Alph}

\begin{center}
{\ \vspace{3.4cm}}

\begin{minipage}[t][2.8cm][s]{\textwidth}%
\centering
\thesistitlefontHUGE\sffamily\bfseries\tuinfthesistitle\\
\bigskip
{\thesistitlefonthuge\sffamily\bfseries\tuinfthesissubtitle}
\end{minipage}

\vspace{1.3cm}

{\thesistitlefontLARGE\sffamily \tuinfthesistype}

\vspace{6mm}

{\thesistitlefontlarge\sffamily zur Erlangung des akademischen Grades}

\vspace{6mm}

{\thesistitlefontLARGE\sffamily\bfseries \tuinfthesisdegree}

\vspace{6mm}

{\thesistitlefontlarge\sffamily im Rahmen des Studiums}

\vspace{6mm}

{\thesistitlefontLarge\sffamily\bfseries \tuinfthesiscurriculum}

\vspace{6.5mm}

{\thesistitlefontlarge\sffamily eingereicht von}

\vspace{6mm}

{\thesistitlefontLarge\sffamily\bfseries \tuinfthesisauthor}

\vspace{1.5mm}

{\thesistitlefontlarge\sffamily Matrikelnummer \tuinfthesismatrikelno} 

\vspace{1.4cm}

\vspace{0pt}\raggedright\thesistitlefontnormalsize\sffamily
\begin{minipage}[t][1.6cm][t]{\textwidth}%
  %
  an der

  Fakult\"{a}t f\"{u}r Informatik der Technischen Universit\"{a}t Wien
\end{minipage}

\begin{minipage}[t][4cm][t]{\textwidth}%
  \vspace{0pt}\raggedright\thesistitlefontnormalsize\sffamily
  %
  \begin{tabbing}%
	    \hspace{19mm} \= \hspace{66mm} \kill
	    \tuinfthesisbetreuung: \> \tuinfthesisbetreins\\
	    Mitwirkung: \> \tuinfthesisbetrzwei\\
	                \> \tuinfthesisbetrdrei
  \end{tabbing}
\end{minipage}

\begin{minipage}[t][1.5cm][t]{\textwidth}%
  \vspace{0pt}\sffamily\thesistitlefontnormalsize
  \begin{tabbing}%
    \hspace{45mm} \= \hspace{63mm} \= \hspace{51mm} \kill
    Wien, \tuinfthesisdate \> {\raggedright\rule{51mm}{0.5pt}} \> {\raggedright\rule{51mm}{0.5pt}} \\
    \> \begin{minipage}[t][0.5cm][t]{51mm}\centering (Unterschrift \tuinfthesisverfassung)\end{minipage}
    \> \begin{minipage}[t][0.5cm][t]{51mm}\centering (Unterschrift \tuinfthesisbetreuung)\end{minipage}
    \end{tabbing}
\end{minipage}

\end{center}

% we want an empty page right after first titlepage
\pagestyle{empty}
\cleardoublepage

% we're done with the titlepages, proceed with default pagenumbering
\pagenumbering{roman}

% restore baselineskip
\setlength{\baselineskip}{\tmpbaselineskip}
\setlength{\parindent}{\tmpparindent}

% back to normal geometry
\restoregeometry

\selectlanguage{english}

%%% Local Variables:
%%% TeX-PDF-mode: t
%%% TeX-debug-bad-boxes: t
%%% TeX-parse-self: t
%%% TeX-auto-save: t
%%% reftex-plug-into-AUCTeX: t
%%% End:


% an english translation may follow
% $Id: titlepage.tex 1752 2010-03-20 11:07:02Z tkren $
%
% TU Wien - Faculty of Informatics
% thesis titlepage
%
% This titlepage is using the geometry package, see
% <http://www.ctan.org/macros/latex/contrib/geometry/geometry.pdf>
%
% For questions and comments send an email to
% Thomas Krennwallner <tkren@kr.tuwien.ac.at>
% or to Petra Brosch <brosch@big.tuwien.ac.at>
%

% setup page dimensions for titlepage
\newgeometry{left=2.4cm,right=2.4cm,bottom=2.5cm,top=2cm}

% force baselineskip and parindent
%\newlength{\tmpbaselineskip}
%\setlength{\tmpbaselineskip}{\baselineskip}
%\setlength{\baselineskip}{13.6pt}
%\newlength{\tmpparindent}
%\setlength{\tmpparindent}{\parindent}
%\setlength{\parindent}{17pt}

% first titlepage
\thispagestyle{tuinftitlepage}

%
% Kludge: for each titlepage set \pagenumbering to a different
% style. This is used to fix a problem with hyperref, because there
% are multiple "page 1" and hyperref hates that
%
\pagenumbering{Roman}

\begin{center}
{\ \vspace{3.4cm}}

\begin{minipage}[t][2.8cm][s]{\textwidth}%
\centering
\thesistitlefontHUGE\sffamily\bfseries\tuinfthesistitle\\
\bigskip
{\thesistitlefonthuge\sffamily\bfseries\tuinfthesissubtitle}
\end{minipage}

\vspace{1.3cm}

{\thesistitlefontLARGE\sffamily \tuinfthesistypeen}

\vspace{6mm}

{\thesistitlefontlarge\sffamily submitted in partial fulfillment of the requirements for the degree of}

\vspace{6mm}

{\thesistitlefontLARGE\sffamily\bfseries \tuinfthesisdegreeen}

\vspace{6mm}

{\thesistitlefontlarge\sffamily in}

\vspace{6mm}

{\thesistitlefontLarge\sffamily\bfseries \tuinfthesiscurriculumen}

\vspace{6.5mm}

{\thesistitlefontlarge\sffamily by}

\vspace{6mm}

{\thesistitlefontLarge\sffamily\bfseries \tuinfthesisauthor}

\vspace{1.5mm}

{\thesistitlefontlarge\sffamily Registration Number \tuinfthesismatrikelno} 

\vspace{1.4cm}

\begin{minipage}[t][1.6cm][t]{\textwidth}%
  \vspace{0pt}\raggedright\thesistitlefontnormalsize\sffamily
  %
  to the Faculty of Informatics 

  at the Vienna University of Technology
\end{minipage}

\vspace{0pt}\raggedright\thesistitlefontnormalsize\sffamily
\begin{minipage}[t][4cm][t]{\textwidth}%
  \begin{tabbing}%
	    \hspace{19mm} \= \hspace{66mm} \kill
	    Advisor: \> \tuinfthesisbetreins\\
	    Assistance: \> \tuinfthesisbetrzwei\\
	                \> \tuinfthesisbetrdrei
     \end{tabbing}
\end{minipage}

\begin{minipage}[t][1.5cm][t]{\textwidth}%
  \vspace{0pt}\sffamily\thesistitlefontnormalsize
  \begin{tabbing}%
    \hspace{45mm} \= \hspace{63mm} \= \hspace{51mm} \kill
    Vienna, \tuinfthesisdate \> {\raggedright\rule{51mm}{0.5pt}} \> {\raggedright\rule{51mm}{0.5pt}} \\
    \> \begin{minipage}[t][0.5cm][t]{51mm}\centering (Signature of Author)\end{minipage}
    \> \begin{minipage}[t][0.5cm][t]{51mm}\centering (Signature of Advisor)\end{minipage}
    \end{tabbing}
\end{minipage}

\end{center}

% we want an empty page right after first titlepage
\pagestyle{empty}
\cleardoublepage

% we're done with the titlepages, proceed with default pagenumbering
\pagenumbering{roman}

% restore baselineskip
\setlength{\baselineskip}{\tmpbaselineskip}
\setlength{\parindent}{\tmpparindent}

% back to normal geometry
\restoregeometry


%%% Local Variables:
%%% TeX-PDF-mode: t
%%% TeX-debug-bad-boxes: t
%%% TeX-parse-self: t
%%% TeX-auto-save: t
%%% reftex-plug-into-AUCTeX: t
%%% End:
 % optional

%%%%%%%%%%%%%%%%%%%%%%%%%%%%%%%%%%%%%%%%%
%%%   ERKLAERUNG DER SELBSTAENDIGKEIT   %
%%%%%%%%%%%%%%%%%%%%%%%%%%%%%%%%%%%%%%%%%
\cleardoublepage
\selectlanguage{ngerman}
\chapter*{Erklärung zur Verfassung der Arbeit}

\tuinfthesisauthor\\
\tuinfthesisauthoraddress

\vspace*{1.2cm}

Hiermit erkläre ich, dass ich diese Arbeit selbständig verfasst habe, 
dass ich die verwendeten Quellen und Hilfsmittel vollständig angegeben 
habe und dass ich die Stellen der Arbeit - einschließlich Tabellen, 
Karten und Abbildungen -, die anderen Werken oder dem Internet im 
Wortlaut oder dem Sinn nach entnommen sind, auf jeden Fall unter Angabe 
der Quelle als Entlehnung kenntlich gemacht habe.\\

\vspace*{2cm}
\begin{tabbing}%
    \hspace{58mm} \= \hspace{28mm} \= \hspace{58mm} \kill
    {\raggedright\rule{58mm}{0.5pt}} \> \> {\raggedright\rule{58mm}{0.5pt}} \\
    \begin{minipage}[t][0.5cm][t]{58mm}
	\vspace{0pt}\sffamily\thesistitlefontnormalsize
	\centering (Ort, Datum)
    \end{minipage}
    \> \>
    \begin{minipage}[t][0.5cm][t]{58mm}
	\vspace{0pt}\sffamily\thesistitlefontnormalsize
	\centering (Unterschrift \tuinfthesisverfassung)
    \end{minipage}
\end{tabbing}


\selectlanguage{english}

%%%%%%%%%%%%%%%%%%%%%%%%%%%%%%%%%%%%%%%%%
%%%   ACKNOWLEDGEMENTS    %%%%%%%%%%%%%%%
%%%%%%%%%%%%%%%%%%%%%%%%%%%%%%%%%%%%%%%%%

% optional acknowledgements may be included in german or in english
%\chapter*{Danksagung}

Hier fügen Sie optional eine Danksagung ein.
 		% optional
\chapter*{Acknowledgements}

Optional acknowledgements may be inserted here.	% optional

%%%%%%%%%%%%%%%%%%%%%%%%%%%%%%%%%%%%%%%%%
%%%   ABSTARCT    %%%%%%%%%%%%%%%%%%%%%%%
%%%%%%%%%%%%%%%%%%%%%%%%%%%%%%%%%%%%%%%%%

\chapter*{Abstract}

According to the guidelines of the faculty, an abstract in English has to be inserted here.
\cleardoublepage
\selectlanguage{ngerman}
\chapter*{Kurzfassung}

Hier fügen Sie die Kurzfassung auf Deutsch gemäß den Vorgaben der Fakultät ein.

\selectlanguage{english}

%%%%%%%%%%%%%%%%%%%%%%%%%%%%%%%%%%%%%%%%%
%%%   CONTENTS    %%%%%%%%%%%%%%%%%%%%%%%
%%%%%%%%%%%%%%%%%%%%%%%%%%%%%%%%%%%%%%%%%
% uncomment to set document language to german (results in "Inhaltsverzeichnis",
% "Kapitel", "Abbildung", etc. instead of "Contents", "Chapter", and "Figure"),
% otherwise the document's language is english
%\selectlanguage{ngerman}

\setcounter{tocdepth}{1}

\cleardoublepage
\pagestyle{numberCorner}
\tableofcontents*

%%%%%%%%%%%%%%%%%%%%%%%%%%%%%%%%%%%%%%%%%
%%%   MAINMATTER    %%%%%%%%%%%%%%%%%%%%%
%%%%%%%%%%%%%%%%%%%%%%%%%%%%%%%%%%%%%%%%%

\mainmatter
% Include number also in subsubsections
\setsecnumdepth{subsubsection}

\pagenumbering{arabic}
\pagestyle{numberCorner}

%%%%%%%%%%%%%%%%%%%%%%%%%%%%%%%%%%%%%%%%%
\cfchapter{Introduction}{chapters}{introduction}
%%%%%%%%%%%%%%%%%%%%%%%%%%%%%%%%%%%%%%%%%

%%%%%%%%%%%%%%%%%%%%%%%%%%%%%%%%%%%%%%%%%
\cfchapter{Typographic Design}{chapters}{design}
%%%%%%%%%%%%%%%%%%%%%%%%%%%%%%%%%%%%%%%%%

%%%%%%%%%%%%%%%%%%%%%%%%%%%%%%%%%%%%%%%%%
\cfchapter{Bibliographic Issues}{chapters}{bibliographic}
%%%%%%%%%%%%%%%%%%%%%%%%%%%%%%%%%%%%%%%%%

%%%%%%%%%%%%%%%%%%%%%%%%%%%%%%%%%%%%%%%%%
\cfchapter{Updated Features}{chapters}{update}
%%%%%%%%%%%%%%%%%%%%%%%%%%%%%%%%%%%%%%%%%

%%%%%%%%%%%%%%%%%%%%%%%%%%%%%%%%%%%%%%%%%
%%% BACKMATTER %%%%%%%%%%%%%%%%%%%%%%%%%%
%%%%%%%%%%%%%%%%%%%%%%%%%%%%%%%%%%%%%%%%%

\appendix

\bibliographystyle{IEEEtranS}
\bibliography{references}

\printnomenclature

\end{document}
