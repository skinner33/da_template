This is a short introduction into the features added.

\cfsection{References}{subdir}{refs}

\cfsection{Subfigure}{subdir}{subfig}

\section{Other Stuff}

\emph{siunitx} might not be used much in the computer science, but a bit
frequency calculation might be nice:
\num{2d10} \si{\hertz} = \num{20} \si{\giga\hertz} = \num{2d10}
\si{\per\second}=  \num{2d-10} \si{\second}

Or some data size calculations:
\begin{math}
	\SI{100}{\mebi\byte} \neq \SI{100}{\mega\byte}
\end{math}

Surprisingly there are some packages missing which I think are commonly used in
computer science, i.e.\ \emph{dsfont} for $i \in \mathds{N}$ or \emph{amsmath}
for $3 \equiv 6 \mod{3}$.

\begin{lem} \label{lem:l1}
	Some theses will include lemmas and proofs
\end{lem}
\begin{lem} \label{lem:l2}
	$1 + 1 = 2$
\end{lem}
\begin{thm} \label{thm:t1}
	A theorem
\end{thm}
\begin{proof}
	Isn't it obvious?
\end{proof}
\Cref{lem:l1,lem:l2} need no proof.


Also very fancy and useful it the \emph{todonotes} package\todo{i bet you
	didn't missed this todo :)}

Oh, and do not forget that there is an acronym-package.
Please, remove this \ac{TMR} in you text and use this package instead of
	typing it every time.
This also allows you to add a nice list of the used acronyms, which is not
	that important.

Similar to acronyms, you can use the nomencl-package to define you
	nomenclature, which may be of more interest in some cases.
\nomenclature[1a]{$\pi$}{A factor which is relevant to pizza!}
For example, we now say the $\pi$ will denote something related to pizza!
Which is now listed on the very last page, behind the bibliographic, so that
	everybody will miss it.

As usual, consult the documentation of the respective package for more
	informations.

\section{Non \LaTeX{} stuff}
You might want to checkout awc
\url{https://github.com/devd/Academic-Writing-Check} for some basic checking,
mainly the typographic stuff related to \LaTeX{} is nice.

Also \texttt{aspell} has a special mode for checking \LaTeX{} and \TeX{}
documents: \texttt{aspell --mode=tex -c thesis.tex}.\footnote{You have to call
	aspell for every tex file, included files are \textbf{not} checked!}
In this mode most \TeX{ commands are ignored (but not all).

There is also a great amount of other tools for checking style or grammar,
but most of them don't process \TeX{} (and i haven't used them yet, so there
may be dragons).
You can use \texttt{detex} to get plain-text from your \TeX{} files, but it has
its limitations.

With the plain-text you can use GNU programs \texttt{diction} and
\texttt{style},\footnote{\url{https://www.gnu.org/software/diction/diction.html}}
or something more advanced like \textit{after the deadline}
\footnote{\url{http://www.afterthedeadline.com/}} or \textit{languageTool}.
\footnote{\url{http://www.languagetool.org/}}


Another hint:
If you use a VCS (and you should!\footnote{git \url{http://git-scm.com/} is my
	personal favorite}) then you may want to start a new line after every .\
and keep the length of your lines at about 80 characters (word-wrap).
The former is personal conditioning, the later is supported by most editors
and normally done anyway (but not always to 80).

You might want to know why this is useful:
If you use a VCS you might want to see the difference between two versions.
This is supported by every VCS, but if you write a whole page of text as one
monolithic block, then one new word at the wrong place can cause a
rewrapping of the whole thing (especially if the line length your editor uses
depends on its windowsize).
Then you still have a diff, but i doubt it will show you the real difference.
