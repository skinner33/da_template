\label{sec:ref}

This is just to show how to make a separation of the tex documents.

\subsection{And a subsection}
\label{ssec:subsec}

How nice is it to write here?

\subsubsection{And a subsubsection}
\label{sssec:subsubsec}

Well, same as in \cref{ssec:subsec}.
So basically \cref{sec:ref,ch:bibliographic,sssec:subsubsec,ssec:subsec} have
no content what so ever (ok, the chapter reference is just there to show how
cool cref is).

As will be shown later, figures and others stuff can also be referenced very
conveniently.
If every reference, as \Cref{sfig:a}, should be capitalized, then passing the
option \texttt{capitalise} to the cleveref package allows this.
Passing the option \texttt{nameinlink} also includes the environment name into
the linked area.

As \emph{cleveref} is very powerful:
The following section is produced by the cref using the given label:
\namecref{ssec:subsec}.
This is also used in every normal references, thus there is no need for custom
commands which add the name of the environment.
Although for some environment, e.g.\ some algorithm packages, you need to
define the environment name.

Consulting the documentation of cref might be a good idea if you dislike
something about the way the references are handled or formatted.

BTW: did you notice that cref sorts the references?
In the second sentence of this section the chapter reference was the second
reference given, but in the output it is the last.

